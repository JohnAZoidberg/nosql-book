% Reference abbreviation entry, do not remove in PR
\newabbreviation{foobar}{FOOBAR}{fully operational organization building acronym references}


\newacronym{sql}{SQL}{Structured Query Language}
\newacronym{cql}{CQL}{Cassandra Query Language}
\newglossaryentry{nosql}{
  name={NoSQL},
  description={
    \textit{No SQL} or \textit{Not only SQL}.
    NoSQL is a loosely defined term, grouping different
    databases which either do not only support data
    accesses via the Structured Query Language (SQL),
    or do not support it at all.
  },
}

\newacronym{crud}{CRUD}{Create, Read, Update, Delete}
\newacronym{api}{API}{Application Programming Interface}
\newacronym{mdm}{MDM}{Master Data Management}
\newacronym{it}{IT}{Information Technology}
\newacronym{orm}{ORM}{object-relational mapping}
\newacronym{ogm}{OGM}{object-graph mapping}
\newacronym{rdbms}{RDBMS}{relational database management system}
\newacronym{iam}{IAM}{identity and access management}
\newacronym{oltp}{OLTP}{online transaction processing}
\newacronym{acid}{ACID}{Atomicity, Consistency, Isolation, Durability}

\newglossaryentry{cern}{
  name={CERN},
  description={
    CERN, the European Organization for Nuclear Research, is one of the world's
    largest and most respected centres for scientific research.
  },
}
\newglossaryentry{atlas}{
  name={ATLAS},
  description={
   ATLAS (A Toroidal LHC ApparatuS) is one of the seven particle detector
   experiments at the Large Hadron Collider (LHC), a particle accelerator at CERN.
   It generates 1 petabyte of raw data per second, even after filtering it
   still requires over 100 megabytes of disk space per second – at least a
   petabyte each year \autocite{fermi2004giant}.
  },
}
